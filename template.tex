%%%%%%%%%%%%%%%%%%%%%%%%%%%%%%%%%%%%%%%%%%%%%%%%%%%%%%%%%%%%%%%%%%
%%% 語研論集用テンプレート (pLaTeX2eのみ)
%%% 2020/11/17
%%% 野元 裕樹
%%%%%%%%%%%%%%%%%%%%%%%%%%%%%%%%%%%%%%%%%%%%%%%%%%%%%%%%%%%%%%%%%%
\documentclass{goken}
%タイトル
\title{『語研論集』\LaTeX{}テンプレート}
%副題
\subtitle{副題(任意)}
%タイトル(英)
\entitle{\LaTeX{} template for \textit{Goken Ronshu}}
%副題(英)
\ensubtitle{English subtitle (if any)}

%著者名(複数の場合は「,」で区切る)
\author{野元 裕樹}
%著者名(ヘッダー用;スペースなし;複数の場合は全角の「,」で区切る;一行に収まらない場合は「他」を用いる)
\hauthor{野元裕樹}
%著者名(英)(複数の場合は「, 」で区切る)
\enauthor{Hiroki Nomoto}
%著者名(英;ヘッダー用;複数の場合は「, 」で区切る;一行に収まらない場合は「et al.」を用いる)
\enhauthor{Hiroki Nomoto}

%所属(複数の場合もすべてここに入力)
\affil{東京外国語大学大学院総合国際学研究院\\
School of Language and Culture Studies, Tokyo University of Foreign Studies%
}

%カテゴリー
%\category{論文}
%\category{研究ノート}
\category{特集「\LaTeX{}で書く」}%特集名を記入する
%\category{特集補遺「XXX」}%特集名を記入する
%\category{書評}
%\category{翻訳}

%執筆者連絡先
\email{nomoto@tufs.ac.jp}

%原稿受領日(提出日に変更する)
\received{2020年11月17日}

%発行年
\date{2020}
%号
\jourvolume{25}
%開始ページ(ページ番号確定後に変更する)
\startpage{1}
%終了ページ(ページ番号確定後に変更する)
\endpage{3}


%%%% 各自必要なパッケージを以下に追加 %%%%%
\usepackage{url}
\usepackage[normalem]{ulem}
\usepackage{amsmath}
\usepackage{tipa}
\usepackage{booktabs}

%%%%%%%%%%%%%%%%%%%%%%%%%%%%%%%%%

\begin{document}

\maketitle

%要旨
\begin{abstract}
	この文章は『語研論集』の原稿を\LaTeX{}で書くためのテンプレートである.要旨は400字以内.
\end{abstract}
%要旨(英)
\begin{enabstract}
	This document is a \LaTeX{} template for \textit{Goken Ronshu}.
\end{enabstract}

%キーワード
\keyword{キーワード(5つまで)}
%キーワード(英)
\enkeyword{keywords}

\section{はじめに}
『語研論集』原稿執筆において注意すべき点は,句読点は「、」「。」でなく,「,」「.」を用いることである\footnote{和文では注番号は句読点の前に打つ.}.
例文・図表の様式などを含むその他の項目には,特に規定はない.

\newpage
%偶数ページ
\section{簡単な例}
表\ref{tab:cite}は,参考文献に登場する文献の種類をまとめたものである.

\begin{table}
	\caption{参考文献中の文献とその種類}
	\begin{tabular}{ll}
		\toprule
		\multicolumn{1}{c}{文献} & \multicolumn{1}{c}{種類}\\
		\midrule
		\citet{AsherLascarides03} & 欧文・書籍\\
		\citet{LatrouiteRiester18} & 欧文・書籍中の章\\
		\citet{NomotoKartini12} & 欧文・雑誌論文\\
		\citet*{宗宮他18} & 和文・書籍\\
		\citet{田窪97a} & 和文・書籍中の章\\
		\citet{吉枝13} & 和文・雑誌論文\\
		\bottomrule
	\end{tabular}
	\label{tab:cite}
\end{table}

\newpage
%奇数ページ
\currentpdfbookmark{参考文献}{参考文献}
%\bibliography{reference}
%unified.bstが整形してくれるが、日本語を中心に適宜修正が必要。

\begin{thebibliography}{6}
\providecommand{\natexlab}[1]{#1}
\providecommand{\url}[1]{#1}
\providecommand{\urlprefix}{}
\expandafter\ifx\csname urlstyle\endcsname\relax
  \providecommand{\doi}[1]{doi:\discretionary{}{}{}#1}\else
  \providecommand{\doi}{doi:\discretionary{}{}{}\begingroup
  \urlstyle{rm}\Url}\fi

\bibitem[{Asher \& Lascarides(2003)}]{AsherLascarides03}
Asher, Nicholas \& Alex Lascarides. 2003.
\newblock \emph{Logics of conversation}.
\newblock Cambridge: Cambridge University Press.

\bibitem[{Latrouite \& Riester(2018)}]{LatrouiteRiester18}
Latrouite, Anja \& Arndt Riester. 2018.
\newblock The role of information structure for morphosyntactic choices in
  {Tagalog}.
\newblock In Sonja Riesberg, Asako Shiohara \& Atsuko Utsumi (eds.),
  \emph{Perspectives on information structure in {Austronesian} languages},
  247--284. Berlin: Language Science Press.
\newblock \doi{10.5281/zenodo.1402549}.

\bibitem[{Nomoto \& Kartini(2012)}]{NomotoKartini12}
Nomoto, Hiroki \& Kartini Abd.~Wahab. 2012.
\newblock \textit{Kena} adversative passives in {M}alay, funny control, and
  covert voice alternation.
\newblock \emph{Oceanic Linguistics} 51(2). 360--386.
\newblock \doi{10.1353/ol.2012.0017}.

\bibitem[{宗宮他(2018)宗宮, 糸川 \& 野元}]{宗宮他18}
宗宮喜代子, 糸川健 \& 野元裕樹. 2018.
\newblock 『動詞の「時制」がよくわかる英文法談義』
\newblock 大修館書店.

\bibitem[{田窪(1997)}]{田窪97a}
田窪行則. 1997.
\newblock 日本語の人称表現.
\newblock 田窪行則(編)『視点と言語行動』, 13--41. くろしお出版.

\bibitem[{吉枝(2013)}]{吉枝13}
吉枝聡子. 2013.
\newblock 「ペルシア語の所有・存在表現」
\newblock 『語学研究所論集』18. 362--378.\\
  \urlprefix\url{http://hdl.handle.net/10108/76217}.

\end{thebibliography}
\metainfo
\end{document}
